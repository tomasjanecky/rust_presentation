\documentclass[10pt]{beamer}
\usepackage[utf8]{inputenc}
\usepackage{xeCJK}
\usepackage{graphicx}
\usepackage {mathtools}
\usepackage{utopia} 
\usepackage{listings}
\usetheme{CambridgeUS}
\usecolortheme{dolphin}
\usepackage[UKenglish]{isodate}

%rust colors https://www.reddit.com/r/rust/comments/f7ocdx/rust_the_bookstyle_syntax_highlighting_for_latex/
\usepackage{color}
\usepackage{listings}
\definecolor{GrayCodeBlock}{RGB}{241,241,241}
\definecolor{BlackText}{RGB}{110,107,94}
\definecolor{RedTypename}{RGB}{182,86,17}
\definecolor{GreenString}{RGB}{96,172,57}
\definecolor{PurpleKeyword}{RGB}{184,84,212}
\definecolor{GrayComment}{RGB}{170,170,170}
\definecolor{GoldDocumentation}{RGB}{180,165,45}
\lstdefinelanguage{rust}
{
    columns=fullflexible,
    keepspaces=true,
    %backgroundcolor=\color{GrayCodeBlock},
    basicstyle=\ttfamily\color{BlackText},
    keywords={
        true,false,
        unsafe,async,await,move,
        use,pub,crate,super,self,mod,
        struct,enum,fn,const,static,let,mut,ref,type,impl,dyn,trait,where,as,
        break,continue,if,else,while,for,loop,match,return,yield,in
    },
    keywordstyle=\color{PurpleKeyword},
    ndkeywords={
        bool,u8,u16,u32,u64,u128,i8,i16,i32,i64,i128,char,str,
        Self,Option,Some,None,Result,Ok,Err,String,Box,Vec,Rc,Arc,Cell,RefCell,HashMap,BTreeMap,
        macro_rules
    },
    ndkeywordstyle=\color{RedTypename},
    comment=[l][\color{GrayComment}\slshape]{//},
    morecomment=[s][\color{GrayComment}\slshape]{/*}{*/},
    morecomment=[l][\color{GoldDocumentation}\slshape]{///},
    morecomment=[s][\color{GoldDocumentation}\slshape]{/*!}{*/},
    morecomment=[l][\color{GoldDocumentation}\slshape]{//!},
    morecomment=[s][\color{RedTypename}]{\#![}{]},
    morecomment=[s][\color{RedTypename}]{\#[}{]},
    stringstyle=\color{GreenString},
    string=[b]"
}

\lstset{language=C++,
                basicstyle=\ttfamily,
                keywordstyle=\color{blue}\ttfamily,
                stringstyle=\color{red}\ttfamily,
                commentstyle=\color{green}\ttfamily,
                %backgroundcolor=\color{GrayCodeBlock},
                morecomment=[l][\color{magenta}]{\#}
}


\definecolor{myNewColorA}{RGB}{0, 57, 93}%https://www.designpieces.com/palette/barclays-color-palette-hex-and-rgb/
\definecolor{myNewColorB}{RGB}{0, 57, 93}
\definecolor{myNewColorC}{RGB}{0, 57, 93}
\setbeamercolor*{palette primary}{bg=myNewColorC}
\setbeamercolor*{palette secondary}{bg=myNewColorB, fg = white}
\setbeamercolor*{palette tertiary}{bg=myNewColorA, fg = white}
\setbeamercolor*{titlelike}{fg=myNewColorA}
\setbeamercolor*{title}{bg=myNewColorA, fg = white}
\setbeamercolor*{item}{fg=myNewColorA}
\setbeamercolor*{caption name}{fg=myNewColorA}
\usefonttheme{professionalfonts}
\usepackage{natbib}
\usepackage{hyperref}
%------------------------------------------------------------
\titlegraphic{\includegraphics[height=1.5cm]{logo.png}} 

\setbeamerfont{title}{size=\large}
\setbeamerfont{subtitle}{size=\small}
\setbeamerfont{author}{size=\small}
\setbeamerfont{date}{size=\small}
\setbeamerfont{institute}{size=\small}
\title[Barclays]{Comparison of Rust and Java}

\author[Tomáš Janecký Michael Vlach]{Tomáš Janecký Michael Vlach}

\institute[Barclays]{BARX FX}
\date[\textcolor{white}{\today} ]
{\today}



\begin{document}

\frame{\titlepage}


\section{Introduction}
    \begin{frame}{About Rust}
	    \begin{itemize}
	        \item introduced in 2010
	        \item compiler guaranties memory, thread and type safety by default (can be violated with \texttt{unsafe\{\}} block)
	        \item Haskell inspired polymorphism
	        \item memory is freed automatically without garbage collector 
	        \item no concept of Null pointer, \texttt{Option} type instead (similar to \texttt{Optional<T>})
	    \end{itemize}
	    
	
    \end{frame}
    \begin{frame}{Rust users}
    \begin{itemize}
        \item Will be added to Linux kernel $6.1$
	    \item Microsoft Azure (C/C++ deprecated)
	    \item JP Morgan
	    \item used in Mozilla Firefox
	 \end{itemize}
    \end{frame}

\section{Comparison with Java}

\begin{frame}[fragile]{Memory safety}
	\begin{columns}
        \begin{column}{0.5\textwidth}
            \begin{lstlisting}[language=java]
class Main {
 public static Integer 
                f(Integer x){
        return null;
 }
 public static void 
         main(String[] args){
  Integer x = 10;
  x = f(x);
  System.out.
    println(x.intValue());
 }
}
            \end{lstlisting}
        \end{column}
        
        \vrule{}
        \begin{column}{0.5\textwidth}
            \begin{lstlisting}[language=rust]
fn f(x: &mut i32)->Option<i32>{
  None
}

fn main() {
  let mut x = 10;
  x = *f(&mut x);//has to be 
                 //matched
  println!("{}", x);
}
            \end{lstlisting}
        
        \end{column}
    \end{columns}
    \begin{verbatim}
    error[E0614]: type `Option<i32>` cannot be dereferenced
    --> src/main.rs:7:9
    
        x = *f(&mut x);
            ^^^^^^^^^^
    \end{verbatim}
\end{frame}

\begin{frame}[fragile]{Memory safety 2}
   \begin{columns}
        \begin{column}{0.5\textwidth}
            \begin{lstlisting}[language=java]
class Main {
 public static void 
       main(String[] args){
  int[] numbers = 
       {10, 20, 30};
  System.out.
       println(numbers[3]);
    }
}
            \end{lstlisting}
        \end{column}
        
        \vrule{}
        \begin{column}{0.5\textwidth}
            \begin{lstlisting}[language=rust]
fn main() {
  let numbers = [10, 20, 30];
  println!("{}", numbers[3]);
}
            \end{lstlisting}
        
        \end{column}
    \end{columns}
    \texttt{println!("{}", numbers[3]);\\
   ~~~~~~~~~~~~~\^{}\^{}\^{}\^{}\^{}\^{}\^{}\^{}\^{}\^{} index out of bounds: the length is 3 but the index is 3
}
\end{frame}

\begin{frame}[fragile]{Thread safety}
	\begin{columns}
        \begin{column}{0.5\textwidth}
            \begin{lstlisting}[language=java]
class Main {
 public static void f(
                 int[] x){
  for(int i=0 ; i<100 ; i++){
   x[0] += 1;
  }
 }
 public static void main(
            String[] args) {
  int[] x = new int[1];
  Thread t1 = 
      new Thread(() -> f(x));
  Thread t2 = 
      new Thread(() -> f(x));
  t1.start();
  t2.start();
  try {//join and println...


            \end{lstlisting}
        \end{column}
        
        \vrule{}
        \begin{column}{0.5\textwidth}
            \begin{lstlisting}[language=rust]
use std::{thread, cell::RefCell,
                        rc::Rc};
fn f(x: Rc<RefCell<i32>>){
  for _i in 1..100 {
    *x.borrow_mut() += 1;
  }
}
fn main() {
 let x = 
     Rc::new(RefCell::new(0));
 let t1 = 
     thread::spawn(||{ f(x)});
 let t2 = 
     thread::spawn(||{ f(x)});
 t1.join().unwrap();
 t2.join().unwrap();
 println!("{:?}", x);
}
            \end{lstlisting}
        
        \end{column}
    \end{columns}
\end{frame}

\begin{frame}[fragile]{Thread safety cont.}
\begin{verbatim}

let t1 = thread::spawn(|| { f(x)});
         ^^^^^^^^^^^^^ -- within this 
         |
        `Rc<RefCell<i32>>` cannot be sent between threads safely
note: required because it's used within this closure


     let t1 = thread::spawn(|| { f(x)});
                            ^^
note: required by a bound in `spawn`
\end{verbatim}  
\end{frame}

\section*{Acknowledgement}  

    \begin{frame}{More Information}
        \begin{itemize}
            \item \href{https://doc.rust-lang.org/book/}{The Rust Programming Language - book}
            \item \href{https://developer.okta.com/blog/2022/03/18/programming-security-and-why-rust}{More about safety of Rust}
            \item \href{https://blog.mozilla.org/en/mozilla/mozilla-welcomes-the-rust-foundation/}{Mozilla Rust foundation}
        \end{itemize}
    \end{frame}

    \begin{frame}
    \textcolor{myNewColorA}{\Huge{\centerline{Thank you!}}}
    \end{frame}

\end{document}
