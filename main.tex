\documentclass[10pt]{beamer}
\usepackage[utf8]{inputenc}
\usepackage{xeCJK}
\usepackage{graphicx}
\usepackage {mathtools}
\usepackage{utopia} 
\usepackage{listings}
\usetheme{CambridgeUS}
\usecolortheme{dolphin}
\usepackage[UKenglish]{isodate}

%rust colors https://www.reddit.com/r/rust/comments/f7ocdx/rust_the_bookstyle_syntax_highlighting_for_latex/
\usepackage{color}
\usepackage{listings}
\definecolor{GrayCodeBlock}{RGB}{241,241,241}
\definecolor{BlackText}{RGB}{110,107,94}
\definecolor{RedTypename}{RGB}{182,86,17}
\definecolor{GreenString}{RGB}{96,172,57}
\definecolor{PurpleKeyword}{RGB}{184,84,212}
\definecolor{GrayComment}{RGB}{170,170,170}
\definecolor{GoldDocumentation}{RGB}{180,165,45}
\lstdefinelanguage{rust}
{
    columns=fullflexible,
    keepspaces=true,
    %backgroundcolor=\color{GrayCodeBlock},
    basicstyle=\ttfamily\color{BlackText},
    keywords={
        true,false,
        unsafe,async,await,move,
        use,pub,crate,super,self,mod,
        struct,enum,fn,const,static,let,mut,ref,type,impl,dyn,trait,where,as,
        break,continue,if,else,while,for,loop,match,return,yield,in
    },
    keywordstyle=\color{PurpleKeyword},
    ndkeywords={
        bool,u8,u16,u32,u64,u128,i8,i16,i32,i64,i128,char,str,
        Self,Option,Some,None,Result,Ok,Err,String,Box,Vec,Rc,Arc,Cell,RefCell,HashMap,BTreeMap,
        macro_rules
    },
    ndkeywordstyle=\color{RedTypename},
    comment=[l][\color{GrayComment}\slshape]{//},
    morecomment=[s][\color{GrayComment}\slshape]{/*}{*/},
    morecomment=[l][\color{GoldDocumentation}\slshape]{///},
    morecomment=[s][\color{GoldDocumentation}\slshape]{/*!}{*/},
    morecomment=[l][\color{GoldDocumentation}\slshape]{//!},
    morecomment=[s][\color{RedTypename}]{\#![}{]},
    morecomment=[s][\color{RedTypename}]{\#[}{]},
    stringstyle=\color{GreenString},
    string=[b]"
}

\lstset{language=C++,
                basicstyle=\ttfamily,
                keywordstyle=\color{blue}\ttfamily,
                stringstyle=\color{red}\ttfamily,
                commentstyle=\color{green}\ttfamily,
                %backgroundcolor=\color{GrayCodeBlock},
                morecomment=[l][\color{magenta}]{\#}
}


\definecolor{myNewColorA}{RGB}{0, 57, 93}%https://www.designpieces.com/palette/barclays-color-palette-hex-and-rgb/
\definecolor{myNewColorB}{RGB}{0, 57, 93}
\definecolor{myNewColorC}{RGB}{0, 57, 93}
\setbeamercolor*{palette primary}{bg=myNewColorC}
\setbeamercolor*{palette secondary}{bg=myNewColorB, fg = white}
\setbeamercolor*{palette tertiary}{bg=myNewColorA, fg = white}
\setbeamercolor*{titlelike}{fg=myNewColorA}
\setbeamercolor*{title}{bg=myNewColorA, fg = white}
\setbeamercolor*{item}{fg=myNewColorA}
\setbeamercolor*{caption name}{fg=myNewColorA}
\usefonttheme{professionalfonts}
\usepackage{natbib}
\usepackage{hyperref}
%------------------------------------------------------------
\titlegraphic{\includegraphics[height=1.5cm]{logo.png}} 

\setbeamerfont{title}{size=\large}
\setbeamerfont{subtitle}{size=\small}
\setbeamerfont{author}{size=\small}
\setbeamerfont{date}{size=\small}
\setbeamerfont{institute}{size=\small}
\title[Barclays]{Comparison of Rust and other programming languages}

\author[author]{author}

\institute[Barclays]{department?}
\date[\textcolor{white}{\today} ]
{\today}



\begin{document}

\frame{\titlepage}

\begin{frame}
\frametitle{Contents}
\tableofcontents

\end{frame}

\section{Introduction}
    \begin{frame}{About Rust}
	    \begin{itemize}
	        \item introduced in 2010
	        \item compiler guaranties memory, thread and type safety by default (can be violated with \texttt{unsafe\{\}} block)
	        \item Haskell inspired polymorphism
	        \item memory is freed automatically without garbage collector 
	        \item no concept of Null pointer, \texttt{Option} type instead (similar to \texttt{std::optional} and \texttt{Optional<T>})
	        \item Will be added to Linux kernel $6.1$
	        \item used in Mozilla Firefox
	    \end{itemize}
	    
	
    \end{frame}
    
    \begin{frame}[fragile]{Code example}
        \begin{lstlisting}[language=rust, caption={Example program from The Rust Programming Language}]
    //returns reference to longer string
    fn longest<'a>(x: &'a str, y: &'a str) -> &'a str
    //         ^       ^           ^           ^
    //         lifetime annotations
    {
        if x.len() > y.len() {
            x //return is optional
        } else {
            y //return is optional
        }
    }
        \end{lstlisting}
    \end{frame}

\section{Comparison with other languages}

\begin{frame}[fragile]{Memory safety}
	\begin{columns}
        \begin{column}{0.5\textwidth}
            \begin{lstlisting}[language=c++]
#include <string>
#include <iostream>
int main(){
    std::string *s  = 
        new std::string
            ("Hello world!");
    delete s;
    std::cout << s << '\n';
    return 0;
}
            \end{lstlisting}
        \end{column}
        
        \vrule{}
        \begin{column}{0.5\textwidth}
            \begin{lstlisting}[language=rust]
use std::mem::drop; 

fn main() {
    let s = "Hello world!"
                .to_string();
    drop(s); // explicit "free"
    println!("{}", x);
}
            \end{lstlisting}
        
        \end{column}
    \end{columns}
    \texttt{let s = "Hello world!".to\_string();\\
~~~~- move occurs because `s` has type `String`, which does not implement the `Copy` trait\\
   drop(s); // explicit "free"\\
~~~~~- value moved here\\
   println!("{}", s);\\
~~~~~~~~~~~~ \^{} value borrowed here after move}
\end{frame}

\begin{frame}[fragile]{Memory safety 2}
   \begin{columns}
        \begin{column}{0.5\textwidth}
            \begin{lstlisting}[language=java]
class Main {
 public static void 
       main(String[ ] args) {
  int[] numbers = 
       {10, 20, 30};
  System.out.
       println(numbers[3]);
    }
}
            \end{lstlisting}
        \end{column}
        
        \vrule{}
        \begin{column}{0.5\textwidth}
            \begin{lstlisting}[language=rust]
fn main() {
  let numbers = [10, 20, 30];
  println!("{}", numbers[3]);
}
            \end{lstlisting}
        
        \end{column}
    \end{columns}
    \texttt{println!("{}", numbers[3]);\\
   ~~~~~~~~~~~~~\^{}\^{}\^{}\^{}\^{}\^{}\^{}\^{}\^{}\^{} index out of bounds: the length is 3 but the index is 3
}
\end{frame}

\begin{frame}[fragile]{Thread safety}
	\begin{columns}
        \begin{column}{0.5\textwidth}
            \begin{lstlisting}[language=c++]
            //c++
            \end{lstlisting}
        \end{column}
        
        \vrule{}
        \begin{column}{0.5\textwidth}
            \begin{lstlisting}[language=rust]
            //rust
            \end{lstlisting}
        
        \end{column}
    \end{columns}
    \texttt{rust output}
\end{frame}


\section*{Acknowledgement}  

    \begin{frame}{More Information}
        \begin{itemize}
            \item \href{https://doc.rust-lang.org/book/}{The Rust Programming Language - book}
            \item \href{https://developer.okta.com/blog/2022/03/18/programming-security-and-why-rust}{More about safety of Rust}
            \item \href{https://blog.mozilla.org/en/mozilla/mozilla-welcomes-the-rust-foundation/}{Mozilla Rust foundation}
        \end{itemize}
    \end{frame}

    \begin{frame}
    \textcolor{myNewColorA}{\Huge{\centerline{Thank you!}}}
    \end{frame}

\end{document}




